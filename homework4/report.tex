\documentclass[UTF8]{ctexart}
\usepackage{geometry, CJKutf8}
\geometry{margin=1.5cm, vmargin={0pt,1cm}}
\setlength{\topmargin}{-1cm}
\setlength{\paperheight}{29.7cm}
\setlength{\textheight}{25.3cm}

% useful packages.
\usepackage{amsfonts}
\usepackage{amsmath}
\usepackage{amssymb}
\usepackage{amsthm}
\usepackage{enumerate}
\usepackage{graphicx}
\usepackage{multicol}
\usepackage{fancyhdr}
\usepackage{layout}
\usepackage{listings}
\usepackage{float, caption}

\lstset{
    basicstyle=\ttfamily, basewidth=0.5em
}

% some common command
\newcommand{\dif}{\mathrm{d}}
\newcommand{\avg}[1]{\left\langle #1 \right\rangle}
\newcommand{\difFrac}[2]{\frac{\dif #1}{\dif #2}}
\newcommand{\pdfFrac}[2]{\frac{\partial #1}{\partial #2}}
\newcommand{\OFL}{\mathrm{OFL}}
\newcommand{\UFL}{\mathrm{UFL}}
\newcommand{\fl}{\mathrm{fl}}
\newcommand{\op}{\odot}
\newcommand{\Eabs}{E_{\mathrm{abs}}}
\newcommand{\Erel}{E_{\mathrm{rel}}}

\begin{document}

\pagestyle{fancy}
\fancyhead{}
\lhead{王琰博, 3220105837}
\chead{数据结构与算法第四次作业}
\rhead{Oct.16th, 2024}

\section{测试程序的设计思路}

首先参考.h文件中++算法的实现完成了对--的重载。
关于.c文件的书写。大致思路就是按照.h文件中函数的功能依照顺序进行测试。
首先我创建了一个空链表,调用push函数向其中插入了元素,检验插入函数是否可用。然后再调用front和back验证其输出是否正确。然后再调用pop删除掉对应元素再次检查。随后调用insert函数插入元素后再次打印检查。
然后再检验拷贝构造函数,移动函数。然后检测了链表的初始化功能,通过打印首位元素来检验输出是否正确。然后分别对该链表进行清除操作和清空操作,来检测erase函数和clear函数的正确性。最后测试了迭代器对链表进行正向遍历和反向遍历,来检测++和--是否有正确重载。
\section{测试的结果}

测试结果一切正常。

输出为: 

列表是否为空?true \par
push\_front 和 push\_back 后的大小: 3 \par
前面的元素: 5 \par
后面的元素: 15 \par
pop\_front 后,前面的元素: 10 \par
pop\_back 后,后面的元素: 10 \par
插入 20 到前面后,前面的元素: 20 \par
复制后的列表前面的元素: 20 \par
移动后的列表前面的元素: 20 \par
移动后复制列表的大小: 0 \par
赋值后的列表前面的元素: 20 \par
初始化列表,前面的元素: 30 \par
初始化列表,后面的元素: 50 \par
删除第二个元素后,前面的元素: 30 \par
删除第二个元素后,后面的元素: 50 \par
清空列表后是否为空?true \par
rangeList 中的元素: 1 2 3 4 5  \par
逆向顺序中的元素: 5 4 3 2 1  \par
我用 valgrind 进行测试,发现没有发生内存泄露。

\section{bug报告}

一切正常。

%%% Local Variables: 
%%% mode: latex
%%% TeX-master: t
%%% End: 
\end{document}
